\chapter*{Abstract}

Wrist-worn accelerometers collect continuous data on human movement. In this dissertation, I develop and apply statistical methods to extract information from wrist-worn accelerometry data, with the goals of (1) biometric identification based on walking/gait, and (2) describing population-level gait data from large surveys. The first study (walking fingerprinting) introduces models for subject identification from accelerometry collected during walking. The study uses the joint distribution of the acceleration and its lags in machine learning and functional regression models. The second study scales the walking fingerprinting methods from the to data from the National Health and Nutrition Examination Survey (NHANES), adapting methods for person identification from small sample sizes ($n\leq150$) and controlled walking environments to a free-living, unlabeled sample of over $15{,}000$ individuals. The third study develops the first method for function on scalar regression in complex survey settings, which is crucial for understanding the relationship between functional outcomes, such as steps over the course of the day, and scalar covariates, such as sex, age, and disease status. Together, this work creates a scalable method for ``fingerprinting'' (identifying an individual from their accelerometry-derived walking pattern), and develops methods to enable population-level inference about wearable-device data in large epidemiological studies. These contributions support the development of interpretable, reproducible, and scalable methods for processing and analyzing wearable accelerometry data.