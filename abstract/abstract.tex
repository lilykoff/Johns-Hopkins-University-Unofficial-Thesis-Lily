\chapter*{Abstract}

Wrist-worn accelerometers collect continuous data on human movement. In this dissertation, I develop and apply statistical methods to extract information from wrist-worn accelerometry data, with the goals of (1) biometric identification based on gait, (2) describing stepping and gait patterns, and (3) predicting health outcomes. The first study (walking fingerprinting) introduces models for subject identification based on accelerometry collected during walking by developing machine learning and functional regression prediction frameworks based on the joint distribution of the acceleration and its lags. The second study investigates the implications of deploying different step counting algorithms for wrist accelerometry in the National Health and Nutrition Examination Survey (NHANES), a large, nationally representative US survey, with respect to both population-level inference and mortality prediction. The third study scales the walking fingerprinting methods from the first study to the NHANES, refining methods for person identification in a free-living sample of over $15{,}000$ individuals. Together, this work contributes new methodological tools for analyzing wearable accelerometry data, underscores both the potential and limitations of using step-based metrics in public health research, and creates a scalable method for ``fingerprinting'', or identifying an individual from their accelerometry-derived walking pattern. These contributions support the development of interpretable, reproducible, and scalable methods for wearable accelerometry.