\chapter{Discussion and Conclusion}
\label{chap:conclusion}

Data collected from wearable accelerometers provide continuous measurements of human movement in the real world and therefore have the potential to uncover insights about the relationship between physical activity and health. This thesis advances statistical methodology for analyzing and interpreting data collected from wearable accelerometers, with the goals of improving health monitoring and public health research and providing scalable, open-source tools and software for the community. 

\section{Person identification from accelerometry}

In Chapter~\ref{chap:paper1}, I develop two methods for identifying individuals from their walking pattern as measured by a wearable accelerometer (walking fingerprinting). Both methods leverage the relationship between acceleration and lag acceleration at a sub-second level, which captures differences between individuals in walking dynamics. The methods performed well in small datasets where individuals were known to be walking. In Chapter~\ref{chap:paper2}, I adapted and scaled the method for deployment in the massive NHANES dataset, which is comprised of several days of accelerometry from over $15{,}000$ individuals ($>10$ terabytes of data!). After applying algorithms to identify areas that were likely walking, I used walking fingerprinting methods and was able to correctly identify individuals almost 50\% of the time. I developed an R package \texttt{accelPrint} \cite{accelprint} to enable others to apply my fingerprinting method. 

Walking fingerprinting has implications for clinical and epidemiological research. If walking patterns are unique across individuals, changes in walking patterns could be used to predict early onset of disease, changes in fall risk, or recovery trajectory from an injury; walking patterns could also be used for clustering and better understanding of walking subgroups. Future directions involve exploring these aspects of the walking fingerprint. 

\section{Complex survey function on scalar regression}
In Chapter~\ref{chap:paper3}, I described the first established method for function on scalar regression (FoSR) in complex survey settings. Since accelerometry data are often collected in large epidemiological surveys, accounting for the manner in which individuals were selected into the study is critical for unbiased estimation and valid inference. This method allows for analysis of the relationship between a functional outcome (e.g. steps or physical activity over the course of the day) and scalar covariates of interest. The method is implemented in an R package, \texttt{svyfosr} \cite{svyfosr}. 


\section{Conclusion}
I have developed methods for processing and analyzing raw wearable accelerometry data in order to identify individuals and created a framework for complex survey function on scalar regression. These methods are all accompanied by open source code and R software. This work contributes to methodological and scientific advances in the fields of biostatistics, data science, and public health.
